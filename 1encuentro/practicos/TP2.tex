\documentclass[a4paper,10pt]{article}
\usepackage[utf8]{inputenc}
\usepackage[spanish]{babel}
\usepackage{amsmath}

%opening
\title{{\bf Trabajo Práctico II:}\\ \emph{Monte Carlo}}
\author{P. Pérez}

\footnotetext[1]{Curso: {\emph{Análisis y procesamiento de imágenes radiológicas en el ámbito clínico}}}


\begin{document}

\maketitle

\begin{abstract}
Los ejercicios comprenden el capítulo de aplicaciones de las técnicas de simulación Monte Carlo. Los mismos deben ser entregados en formato PDF o en un notebook de Jupyter especificando código de programación utilizado, funciones implementadas y resultados obtenidos. Se aceptarán trabajos realizados en plataformas Matlab y Python.
\end{abstract}

\section*{Ejercicios}

\begin{enumerate}
 \item Implementar un método basado en Monte Carlo para determinar la probabilidad de obtener un ``doble 6'' lanzando dos dados.
 \item Calcular con método Monte Carlo la semilongitud de onda de una onda sinusoidal.
 \item Calcular con método Monte Carlo la semilongitud de onda de una onda cosenoidal.
 \item Calcular con método Monte Carlo $\int_{-1}^{1}(2 + u^{3/2}) du$.
 \item Determinar la eficiencia de cómputo y realizar un estudio de convergencia para calcular con el método Monte Carlo $\int_{1/4}^{4}\frac{1}{\sqrt{2\pi}}e^{-\frac{x^{2}}{2}}dx$. Estimar la desviación estándar en función del número de muestreo.
 \item Modelar el esperimento de Buffon para estimar el valor de $\pi$, asignando valores que correspondan a los parámetros del problema.
 \item Realizar una simulación Monte Carlo simple para determinar la distancia total $D_{T}$ y distancia neta $D$ (distancia al punto de partida) luego de 10, 100, 1000 y 10000 pasos de 1 unidad de una partícula moviéndose en un plano.
 \item Repetir el ejercicio del item anterior para una partícula moviéndose en 3 dimensiones.
 \item Realizar una simulación Monte Carlo del transporte de partículas en 2D que sólo pueden interactuar de dos modos: absorción o scattering caracterizados por secciones eficaces $\sigma_{A}$ y $\sigma_{S}$ , respectivamente. En particular, se tiene que la distribución angular de $\sigma_{S}$ es isotrópica e independiente de la energía, mientras que $\sigma_{A} = C/E$ , donde $C$ es una constante que satisface la normalización. El problema consiste en calcular la transmisividad de una muestra de espesor L por parte de una haz puntual de partículas de energía inicial $E_{0}$. Fijar valores de absorción (completo depósito local de la energía residual), de modo tal que no se extienda demasiado el tiempo de simulación.
 \end{enumerate}

\end{document}
