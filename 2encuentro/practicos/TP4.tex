\documentclass[a4paper,10pt]{article}
\usepackage[utf8]{inputenc}
\usepackage[spanish]{babel}
\usepackage{amsmath}

%opening
\title{{\bf Trabajo Práctico IV:}\\ \emph{Radiodiagnóstico anatómico}}
\author{P. Pérez}

\footnotetext[1]{Curso: {\emph{Análisis y procesamiento de imágenes radiológicas en el ámbito médico}}}


\begin{document}

\maketitle

\begin{abstract}
Los ejercicios comprenden el capítulo de Procesos Estocásticos. Los mismos deben ser entregados en formato PDF o Jupyter Notebook especificando código de programación utilizado, funciones implementadas y resultados obtenidos. Se aceptarán trabajos realizados en plataformas Matlab y Python.
\end{abstract}

\section*{Ejercicios}

\begin{enumerate}
 \item  Realizar una simulación determinista (considerando solo absorción exponencial) de la formación de una imagen por contraste de absorción de un cubo de agua dede 10 cm de diámetro y de lado irradiado con un haz paralelo de fotones de 50 keV y tamaño de campo de 10 cm $\times$ 10 cm. (Detección ideal). Graficar perfil central de la imagen e interpretarlo en términos de las propiedades físicas. Repetir el cálculo irradiando desde distintos ángulos. Realice la experiencia irradiando desde 10, 50 y 100 ángulos diferentes. Realice una reconstrucción tomográfica en cada caso y compare y discuta los resultados.
 \item Repetir el ejercicio del item anterior para un haz incidente de 2 canales energéticos igualmente probables de 50 keV y 30 keV. Analizar y discutir los resultados obtenidos.
 \item Defina probabilidades de absorción y scattering realistas para un haz de 50 keV y uno de 30 keV y realize una simulación Monte Carlo (ideada por usted) análoga al primer item. Analizar los resultados.
 \item Simular un set up experimental típico del instrumento de laboratorio, considerando fuente, divergencia del haz desde la fuente, colimador principal, muestra (puede ser un fantoma cilíndrico o cúbico) y detector.
 \end{enumerate}


\end{document}
